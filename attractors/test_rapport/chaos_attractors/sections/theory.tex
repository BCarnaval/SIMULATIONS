\section{Théorie} \label{sec: theory}
    Les attracteurs sont des structures fondamentales en théorie du chaos qui jouent un rôle clé dans la compréhension du comportement chaotique des systèmes. Dans cette section, nous allons explorer en détail le concept d'attracteur en théorie du chaos, en examinant les propriétés mathématiques qui les caractérisent. 

\subsection{Attracteurs} \label{subsec: attractors}
    Pour un système dynamique dissipatif donné, l'attracteur est définit par un sous-ensemble d'états dans l'espace des phases vers lesquels la solution au système converge si les conditions initiales de ladite solution sont comprises dans le bassin d'attraction de l'attracteur. Ici, le \textit{bassin d'attraction} représente une zone de l'espace des phases dont les conditions initiales mènent à des trajectoires qui convergent vers un attracteur. Mathématiquement, on dit que le bassin d'attraction $W$ d'un attracteur $A$ est
    \begin{align}
        W(A) = \{r\in R\;\;|\;\lim_{t\to\infty} f(r, t)\in A\},
    \end{align}

    où ici $r$ représente un ensemble de condition initiales appartenant à l'espace des phases $R$ et $f(r, t)$ la trajectoire de la solution. Autrement-dit, l'attracteur est un sous-ensemble de solutions qui permet d'identifier et de prédire une tendance globale dans la trajectoire d'une solution donnée d'un système chaotique, et ce, malgré sa nature imprévisible. \\

    L'attracteur de Lorenz que l'on nommera ici $L$ est l'un des exemples les plus célèbres d'un attracteur chaotique dans la théorie des systèmes dynamiques \cite{lorenz}. Cet attracteur est le fruit d'un système dynamique non linéaire à trois dimensions qui décrit le comportement d'un fluide en mouvement. Mathématiquement, l'attracteur de Lorenz est défini par un ensemble d'équations différentielles ordinaires, qui décrivent l'évolution de trois variables dynamiques $(x, y, z)$ en fonction du temps et des paramètres $\sigma, \rho$ et $\beta$ qui régissent le comportement du système
    \begin{align}
        L = \left\{
        \begin{array}{c}
           \Dot{x} = \sigma(y - x) \\
           \Dot{y} = x(\rho - z) - y \\
           \Dot{z} = xy - \beta z, 
        \end{array}
        \right.
        \label{eq : lorenz}
    \end{align}

    où les paramètres $\sigma, \rho$ et $\beta$ sont respectivement le nombre de Prandtl, le nombre de Rayleigh et un coefficient géométrique quelconque. Un second attracteur initialement proposé par Otto Rössler en 1976 est l'attracteur de Rössler \cite{rossler}. Celui-ci possède une forme particulière et n'a qu'une seule de ses trois équations qui possède un terme non linéaire
    \begin{align}
        R = \left\{
        \begin{array}{c}
           \Dot{x} = -y - z \\
           \Dot{y} = x + ay \\
           \Dot{z} = b + z(x - c), 
        \end{array}
        \right.
        \label{eq : rossler}
    \end{align}

    avec $a, b$ et $c$ des paramètres numériques qui régissent le comportement du système. Le dernier attracteur que nous allons étudier est celui introduit par Bouali en 2013, qui est non seulement par définition très sensible aux conditions initiales, mais également à la source du phénomène unique de chevauchement d'attracteurs \cite{bouali}. Les équations différentielles non-linéaires qui décrivent cet attracteur sont
        \begin{align}
        B = \left\{
        \begin{array}{c}
           \Dot{x} = \alpha x(1 - y) - \beta z \\
           \Dot{y} = -\gamma y(1 - x^2) \\
           \Dot{z} = \mu x,
        \end{array}
        \right.
        \label{eq : bouali}
    \end{align}

    où les paramètres $\alpha, \beta, \gamma$ et $\mu$ sont une fois de plus des coefficients numériques responsables du comportement de l'attracteur.

\subsection{Exposants de Lyapunov} \label{subsec: lyapunov}
    Nous avons définit que les systèmes dynamiques dissipatifs sensibles au changement infinitésimal des conditions initiales sont chaotiques et donc que deux trajectoires peuvent se voir diverger rapidement. Soit deux positions initiales de l'espace des phases $\bm{x}(t)$ et $\bm{x}(t) + \bm{\delta}_0$ tels que montrés sur la figure \ref{fig: theo_lyapunov}
    \begin{figure}[h!]
        \centering
        \includegraphics[scale=0.3]{figs/Orbital_instability_(Lyapunov_exponent).png}
        \caption{Schéma qui exprime l'évolution de la distance entre deux trajectoires $|\bm{\delta}(t)|$ initialement espacées d'un déplacement infinitésimal $|\bm{\delta_0}|$ en fonction du temps \cite{LEs_wiki}.}
        \label{fig: theo_lyapunov}
    \end{figure}
    
    on considère qu'au temps $t=0$ ces trajectoires ont été éloignées du vecteur déplacement $\bm{\delta}_0$ tel que sa norme $|\bm{\delta}_0|$ soit infinitésimale. Dans le contexte de systèmes chaotique, on trouve qu'après un temps d'évolution $t$, la norme du déplacement entre les trajectoires est donné par
    \begin{align}
        |\bm{\delta}(t)| \simeq |\bm{\delta}_0|e^{\lambda t},
        \label{eq : lyapunov_delta}
    \end{align}

    où l'on appelle la quantité $\lambda$ \textit{l'exposant de Lyapunov}. On accède à cette quantité avec quelques manipulation
    \begin{align}
        \lambda \simeq \frac{1}{t}\ln\qty[\frac{|\bm{\delta}(t)|}{|\bm{\delta}_0|}],
    \end{align}

    où pour un système avec pas temporels discrets tels que l'itération $x_{n + 1} = f(x_n)$, se traduit par
    \begin{align}
        \lambda(x_0) = \lim_{n\to\infty}\sum_{i = 0}^{n - 1}\ln(f'(x_i)),
    \end{align}

    avec $x_0$ le point qui correspond à la condition initiale. On voit ici que pour des solutions calculées à l'aide d'algorithmes, il est possible de remplacer la limite itérative infinie par un algorithme de convergence judicieusement choisi tel que \textit{l'algorithme epsilon}. Il est cependant important de noter que jusqu'ici, nous n'avons introduit qu'un seul exposant $\lambda$ (le plus élevé conventionnellement), mais un système de dimension $N$ possède naturellement $N$ exposants de Lyapunov. On appelle ces valeurs le \textit{spectre de Lyapunov} d'un système. Ce spectre possède plusieurs propriétés intéressantes pour l'analyse des systèmes dynamiques \cite{LEs}: \\
    \begin{itemize}
        \item[$\diamond$] Les composantes du spectres sont indépendantes du métrique utilisé pour leur calcul ainsi que du choix des variables. Cela permet de conclure sur l'objectivité et la pertinence du spectre de Lyapunov. \\
        \item[$\diamond$] Si l'exposant le plus élevé du spectre est positif témoigne généralement d'instabilités exponentiellement importantes et consitue une définition de ce que l'on pourrait appeler chaos. \\
        \item[$\diamond$] La somme du spectre de Lyapunov permet de mesurer le taux contraction des volumes de l'espace des phases. Pour des systèmes dissipatifs, une somme $\sum_i\lambda_i < 0$ signifie que les volumes décroissent exponentiellement à 0 alors qu'une somme nulle implique la conservation des volumes de l'espace des phases.
    \end{itemize}