\section{Résultats} \label{sec: resultats}

\subsection{Trajectoires} \label{subsec: res_trajectories}
    Soit les figures \ref{fig: traj_lorenz}, \ref{fig: traj_rossler} et \ref{fig: traj_bouali} qui présentent les résultats pour la simulation des attracteurs (Lorenz, Rössler et Bouali) définis mathématiquement dans la section \fullref{sec: theory}. On voit sur lesdites figures la trajectoires d'une particule de masse unité dans les systèmes dynamiques dissipatifs représentés par les différents attracteurs. Étonnement, les trajectoires montrées sur ces figures ne semble pas du tout chaotiques, et se rapprochent d'arrangements très bien organisés voir même prédictibles. Une fois de plus, la subtilité réside dans la sensibilité de ces systèmes aux conditions initiales. Il est d'ailleurs à noter que ces figures sont des trajectoires qui proviennent d'\textbf{une seule} position initiale à chaque fois. Pour voir à quel point ces attracteurs sont chaotiques ou non, nous devrions exécuter la simulation pour une multitude de coordonnées initiales et ainsi observer la divergence très précoce entre les solutions. Un bon indicateur de signature chaotique est le spectre de Lyapunov et c'est pourquoi nous allons analyser ce spectre en utilisant les mêmes trajectoires attractives. \\

    Il est également intéressant d'observer sur les figures \ref{fig: traj_lorenz}, \ref{fig: traj_rossler} et \ref{fig: traj_bouali} que le théorème d'unicité des solutions d'équations différentielles ordinaires est vérifié \cite{uniqueness}. On observe effectivement aucun croisement dans les trajectoires obtenues.

\subsection{Spectre de Lyapunov} \label{subsec: res_lyapunov}
    Considérons les figures \ref{fig : lyaps_lorenz}, \ref{fig : lyaps_rossler} et \ref{fig : lyaps_bouali}, sur lesquelles ont observe le calcul du spectre de Lyapunov en fonction du temps pour les trajectoires attractives discutées dans la section \fullref{subsec: res_trajectories}. Il est intéressant d'observer la précieuse utilisation de l'algorithme de convergence epsilon introduit dans la section \fullref{subsec: convergence}. La convergence est pertinente et permet bien d'identifier le comportement à long terme ($\lim_{t\to\infty}\lambda_i(t)$) du spectre de Lyapunov pour la trajectoire des attracteurs étudiés. \\

    On remarque premièrement, pour tous les attracteurs, que l'exposant de Lyapunov maximal $\lambda_1$ est positif. Comme introduit dans la section \fullref{subsec: lyapunov}, il s'agit-là d'une signature typique de chaos. Mathématiquement, cela signifie que dans la direction associée à l'exposant $\lambda_1$ deux trajectoires voisines divergent exponentiellement rapidement dans le temps et donc que de toutes petites perturbations peuvent mener à des trajectoires très différentes. On peut ici faire un lien avec la contraction/expansion de l'espace des phases modélisée par l'évolution du volume de la sphère unitaire $U$ vis-à-vis du signe obtenu pour les composantes du spectre de Lyapunov. \\

    Ensuite, on remarque que dans les résultats obtenus, deux des trois exposants sont négatifs. Cela signifie logiquement que deux trajectoires voisines tendent à converger au sein de l'attracteur et demeurer près l'une de l'autre. Physiquement, on peut aussi dire que des exposants négatifs indiquent que les systèmes étudiés sont dissipatifs (perdent de l'énergie au cours du temps) et donc qu'il est normal d'observer la convergence de certaines trajectoires comme par exemple lorsque l'on étudie un oscillateur harmonique amorti. \\

    Finalement, on voit que certains des exposants de Lyapunov sont nuls. Ces derniers n'impliquent pas de comportement chaotique particuliers. On peut conclure que mathématiquement, une perturbation d'une trajectoire ayant un exposant de Lyapunov nul peut se voir diverger mais seulement de façon logarithmique. On appelle ce phénomène: \textit{comportement quasi-périodique}.

\onecolumngrid
\vspace{2cm}

    \begin{figure}[h!]
        \centering
        \includegraphics[scale=0.6]{figs/trajectories/traj_lorenz.png}
        \caption{Trajectoire obtenue pour la simulation d'une particule de masse unité dans le bassin d'attraction de l'attracteur de Lorenz ayant comme position initiale $\bm{r}_0 = (1, 0, -1)$ ainsi que pour un temps de 100 secondes avec un pas de $h = 0.01$.}
        \label{fig: traj_lorenz}
    \end{figure}
    \vspace{1cm}
    \begin{figure}[h!]
        \centering
        \includegraphics[scale=0.6]{figs/trajectories/traj_rossler.png}
        \caption{Trajectoire obtenue pour la simulation d'une particule de masse unité dans le bassin d'attraction de l'attracteur de Rössler ayant comme position initiale $\bm{r}_0 = (1, 1, -1)$ ainsi que pour un temps de 1000 secondes avec un pas de $h = 0.001$.}
        \label{fig: traj_rossler}
    \end{figure}

\clearpage
    
    \begin{figure}[h!]
        \centering
        \includegraphics[scale=0.5]{figs/trajectories/traj_bouali.png}
        \caption{Trajectoire obtenue pour la simulation d'une particule de masse unité dans le bassin d'attraction de l'attracteur de Bouali ayant comme position initiale $\bm{r}_0 = (0.2, 0.2, -0.08)$ ainsi que pour un temps de 1000 secondes avec un pas de $h = 0.001$.}
        \label{fig: traj_bouali}
    \end{figure}

    \begin{figure}[h!]
        \centering
        \begin{minipage}{0.49\textwidth}
          \centering
          \includegraphics[scale = 0.4]{figs/lyapunovs/lyap_lorenz.png}
          \subcaption{}
          \label{fig: lyap_lorenz}
        \end{minipage}
        \begin{minipage}{0.49\textwidth}
          \centering
          \includegraphics[scale = 0.4]{figs/lyapunovs/lyap_lorenz_zoom.png}
          \subcaption{}
          \label{fig: lyap_lorenz_zoom}
        \end{minipage}
        \caption{Spectre de Lyapunov ($\lambda_i\forall i\in\{1, 2, 3\}$) dans la simulation de l'attracteur de Lorenz. (a) Spectre complet. (b) Mise en évidence du comportement pour les exposants $\lambda_1$ et $\lambda_2$ étant donnée la superposition et le bruit.}
        \label{fig : lyaps_lorenz}
    \end{figure}

    \begin{figure}[h!]
        \centering
        \begin{minipage}{0.49\textwidth}
          \centering
          \includegraphics[scale = 0.4]{figs/lyapunovs/lyap_rossler.png}
          \subcaption{}
          \label{fig: lyap_rossler}
        \end{minipage}
        \begin{minipage}{0.49\textwidth}
          \centering
          \includegraphics[scale = 0.4]{figs/lyapunovs/lyap_rossler_zoom.png}
          \subcaption{}
          \label{fig: lyap_rossler_zoom}
        \end{minipage}
        \caption{Spectre de Lyapunov ($\lambda_i\forall i\in\{1, 2, 3\}$) dans la simulation de l'attracteur de Rössler. (a) Spectre complet. (b) Mise en évidence du comportement pour les exposants $\lambda_1$ et $\lambda_2$ étant donnée la superposition et le bruit.}
        \label{fig : lyaps_rossler}
    \end{figure}

    \clearpage

    \begin{figure}[h!]
        \centering
        \begin{minipage}{0.49\textwidth}
          \centering
          \includegraphics[scale = 0.4]{figs/lyapunovs/lyap_bouali.png}
          \subcaption{}
          \label{fig: lyap_bouali}
        \end{minipage}
        \begin{minipage}{0.49\textwidth}
          \centering
          \includegraphics[scale = 0.4]{figs/lyapunovs/lyap_bouali_zoom.png}
          \subcaption{}
          \label{fig: lyap_bouali_zoom}
        \end{minipage}
        \caption{Spectre de Lyapunov ($\lambda_i\forall i\in\{1, 2, 3\}$) dans la simulation de l'attracteur de Bouali. (a) Spectre complet. (b) Mise en évidence du comportement pour les exposants $\lambda_1$, $\lambda_2$ et $\lambda_3$ étant donnée la superposition et le bruit.}
        \label{fig : lyaps_bouali}
    \end{figure}

\twocolumngrid
