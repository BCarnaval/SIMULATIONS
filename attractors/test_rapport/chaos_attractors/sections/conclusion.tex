\section{Conclusion} \label{sec: conclusion}
    Pour conclure, cette étude théorique et numérique de surface au sujet des attracteurs chaotiques a permis de comprendre comment caractériser ces entités topologiques grâce à certaines méthodes numériques et au calcul d'indicateurs comme le spectre de Lyapunov. D'abord en dressant un portrait théorique puis en développant des méthodes numériques telles que la résolution d'équations différentielles, la décomposition de matrices $QR$ ainsi que la convergence de séries à l'aide du langage Python. \\

    Ce processus à permis de vérifier le théorème d'unicité concernant les trajectoires obtenues numériquement. Trajectoires qui constituent les solutions aux systèmes d'équations différentielles qui définissent les attracteurs étudiés. Il a également été possible de conclure sur le comportement des attracteurs par rapport au signe des éléments de leur spectre de Lyapunov par rapport au temps. On a pu observer que les attracteurs chaotiques étudiés avaient bel et bien une signature chaotique au sens de ce spectre ($\lambda_1 > 0$) notamment grâce à la convergence des valeurs des composantes du spectre grâce à l'approximation donnée par l'algorithme epsilon. \\

    Pour poursuivre ce projet, il serait intéressant de prendre d'autres indicateurs tels que les diagrammes de bifurcations de certains systèmes dynamiques afin de voir où apparaissent mathématiquement ces attracteurs dans l'évolution d'un système \cite{bifurcation}. On aurait également pu analyser les points critiques des différents attracteurs.