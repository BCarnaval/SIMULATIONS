\section{Introduction} \label{sec: introduction}

L'intérêt scientifique envers les systèmes chaotiques remonte au 19\up{ème}
siècle alors que Henri Poincaré étudiait le comportement des solutions du
problème à trois corps \cite{poin_carre}. Bien qu'il n'était pas évident que
ce système était de nature chaotique, la forte sensibilité aux conditions
initiales et la non périodicité des solutions étaient toutefois très présentes.
Birkhoff, Kolmogorov et Lorenz sont également d'éminents mathématiciens et
physiciens qui ont plus tard contribué considérablement dans le domaine de la
théorie du chaos respectivement grâce à la théorie ergodique, l'étude des
turbulences/algorithmes et l'étude de la météorologie. C'est d'ailleurs Edward
Lorenz qui en 1963, arrive avec l'idée \textit{l'effet papillon} selon lequel
de légères variations peuvent avoir des conséquences monstrueuses
\cite{butterfly}. Ce n'est cependant qu'au 20\up{ème} siècle que ce sujet
d'étude à connu ses plus grandes victoires et avancées. La raison principale
de cette ascension est la croissance fulgurante de l'ère numérique. L'étude de
tels systèmes est en effet particulièrement coûteuse en ce qui concerne la
partie mathématique puisqu'elle demande une grande quantité d'itérations et
c'est pourquoi les ordinateurs y sont d'une aide précieuse.

L'étude des systèmes dynamiques dissipatifs requiert régulièrement
l'intervention d'outils mathématiques et topologiques complexes pour être en
mesure d'isoler une tendance ou un comportement physique. On entend ici par
\textit{systèmes dynamiques dissipatifs} des systèmes thermodynamiques qui
agissent hors équilibre et dans lesquels les échanges d'énergie et de matière
avec l'environnement sont permis. Ces systèmes sont donc ouverts au sens
thermodynamique et sont généralement à l'origine de l'apparition d'attracteurs.
Plus précisément, les systèmes dynamiques dont la sensibilité aux conditions
initiales est élevée sont aussi appelés systèmes chaotiques. Ceux-ci sont à la
fois déterministes et imprévisibles et c'est ce mélange particulier de
simplicité et de hasard qui constitue le \textit{chaos} Il sera ici question
d'étudier les différentes propriétés et limites de certains attracteurs
fondamentaux qui émergent naturellement des systèmes dynamiques dissipatifs
tels que l'attracteur de Lorenz, Rössler et Bouali. On déterminera leur
sensibilité aux conditions initiales via l'exposant de Lyapunov et la
divergence de différente solutions grâce aux diagrammes de bifurcation. Nous
pourrons ainsi faire un lien entre ces attracteurs et de réels systèmes
physiques dans lesquels ont retrouve ces objets topologiques naturellement.
